\chapter{Hauptteil}

\section{Tabelle}

\begin{center}
\tablehead{ \textbf{Head1} & \textbf{Head2} & \textbf{Head3}
\\ }
\bottomcaption[Beschreibung]{Beschreibung. Quelle: Berger, Vorlesung, 2012, München }
\begin{supertabular}{c|c|c}
\hline
1 & 2 & 3 \\
4 & 5 & 6 \\
7 & 8 & 9 \\
1 & 2 & 3 \\
4 & 5 & 6 \\
7 & 8 & 9 \\
\end{supertabular}
\end{center}

\section{Bilder}

Dies ist ein ganz kurzer Beispieltext \footnote{\cite{Jacobsen2017;S.47}}. Und noch ein weites Zitat \footnote{\cite{Jacobsen2017}}

\begin{figure}
\begin{center}
\includegraphics[scale=0.5]{\TemplatePath/resources/akad_logo.png}
\caption[Akad]{Akad. Quelle: www.akad.de}
\end{center}
\end{figure}

%%Einkommentieren fuer Syntax Highlighting. In vorlage.tex muessen auch 2 Zeilen einkommentiert werden
%\subsection{Syntax Highlighting}
%\begin{figure}[h]
%\begin{minted}[linenos=true,bgcolor=bg]{php}
%<?php 
%$title="Lorem";
%$desc = "Lorem Ipsum";
%include($_SERVER['DOCUMENT_ROOT'].'/header.php'); 
%?>
%\end{minted}
%\caption{Quellcode: Aufruf von header.php (PHP)}
%\label{abb:header}
%\end{figure}

\section{Formeln}

\section{Quellcode}
\begin{figure}
\begin{lstlisting}[language=bash]
echo "Hello World"
\end{lstlisting}
\caption{Bash Hello World}
\end{figure}

