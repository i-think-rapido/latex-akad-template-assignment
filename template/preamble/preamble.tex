
%
%% Document Class (Koma Script) -----------------------------------------
\documentclass[%
  %draft=true,     % draft mode (no images, layout errors shown)
  draft=false,     % final mode 
%%% --- Paper Settings ---
  paper=a4,
  paper=portrait, % landscape
  pagesize=auto, % driver
%%% --- Base Font Size ---
  fontsize=12pt,%
%%% --- Koma Script Version ---
  version=last, %
%%% --- Global Package Options ---
  english,
  ngerman, % language (passed to babel and other packages)
  numbers=noenddot,
  bibliography=totoc
]{scrreprt}

\usepackage{tocbibind} % Anzeigen des Literaturverzeichnisses im TOC https://ctan.kako-dev.de/macros/latex/contrib/tocbibind/tocbibind.pdf
\usepackage{epsfig}
\usepackage{datetime}
\usepackage{babel}
\usepackage{supertabular}
\usepackage{wrapfig}
\usepackage{multirow}
\usepackage[onehalfspacing]{setspace}
\usepackage{listings}
\usepackage{mathptmx}
\usepackage{geometry}
\usepackage{helvet}
\usepackage{courier}
\usepackage{setspace}
\usepackage{textcomp}
\usepackage{fontspec}
\usepackage{indentfirst}
\usepackage[german=quotes]{csquotes} % http://ctan.ebinger.cc/tex-archive/macros/latex/contrib/csquotes/csquotes.pdf
\usepackage[backend=biber,style=akad]{biblatex} % alternative: iso-authoryear
\usepackage{pdfpages}

\IfFileExists{content/literatur.bib}{\addbibresource{content/literatur.bib}}{\addbibresource{content/literatur.bib}}

\lstset{ % Source Code Printer
  backgroundcolor=\color{white},   % choose the background color; you must add \usepackage{color} or \usepackage{xcolor}
  basicstyle=\footnotesize,        % the size of the fonts that are used for the code
  breakatwhitespace=false,         % sets if automatic breaks should only happen at whitespace
  breaklines=true,                 % sets automatic line breaking
  captionpos=b,                    % sets the caption-position to bottom
  deletekeywords={...},            % if you want to delete keywords from the given language
  escapeinside={\%*}{*)},          % if you want to add LaTeX within your code
  extendedchars=true,              % lets you use non-ASCII characters; for 8-bits encodings only, does not work with UTF-8
  keepspaces=true,                 % keeps spaces in text, useful for keeping indentation of code (possibly needs columns=flexible)
  keywordstyle=\color{blue},       % keyword style
  language=Octave,                 % the language of the code
  morekeywords={*,...},            % if you want to add more keywords to the set
  numbers=left,                    % where to put the line-numbers; possible values are (none, left, right)
  numbersep=5pt,                   % how far the line-numbers are from the code
  rulecolor=\color{black},         % if not set, the frame-color may be changed on line-breaks within not-black text (e.g. comments (green here))
  showspaces=false,                % show spaces everywhere adding particular underscores; it overrides 'showstringspaces'
  showstringspaces=true,          % underline spaces within strings only
  showtabs=true,                  % show tabs within strings adding particular underscores
  stepnumber=1,                    % the step between two line-numbers. If it's 1, each line will be numbered
  tabsize=2,                       % sets default tabsize to 2 spaces
  title=\lstname,                   % show the filename of files included with \lstinputlisting; also try caption instead of title
  belowskip= 0pt 
}


%PDF Keywords
\newcommand*{\pdfkeywords}{akad, assignment, meta, information, pdf, hyperref, latex}

%Literaturverzeichnis Titel
\newcommand*{\prefbiblioname}{Literaturverzeichnis}




\makeatother

\geometry{a4paper, left=45mm, right=20mm, top=30mm, bottom=30mm}

\renewcommand*{\chapterheadstartvskip}{\vspace*{0\baselineskip}}

\pagenumbering{roman}


\usepackage[automark,headsepline]{scrlayer-scrpage}

\clearpairofpagestyles
\cfoot[\pagemark]{\pagemark}
\lehead{\headmark}
\rohead{\headmark}

\pagestyle{scrheadings}


% Fuer Schriftart Arial
%\usepackage[scaled]{uarial}

% Installation der Arial Schriftart unter Linux.
% wget http://tug.org/fonts/getnonfreefonts/install-getnonfreefonts
% texlua install-getnonfreefonts
% getnonfreefonts -r
% getnonfreefonts arial-urw


% PDF Einstellungen für Verlinkungen

%PDF Keywords
\newcommand*{\pdfkeywords}{akad, assignment, meta, information, pdf, hyperref, latex}

%Literaturverzeichnis Titel
\newcommand*{\prefbiblioname}{Literaturverzeichnis}

\usepackage[
	pdftitle={\Title},
	pdfsubject={\pdfsubject},
	pdfauthor={\Name},
  pdfkeywords={\pdfkeywords},
  pdfencoding=auto,
	hyperfootnotes=false,
	colorlinks=true,
	linkcolor=black,
	urlcolor=black,
	citecolor=black
]{hyperref} % http://mirrors.ctan.org/macros/latex/contrib/hyperref/doc/manual.html

%%% Abkürzungsverzeichnis
%\usepackage[
%%	footnote,	% Full names appear in the footnote
%%	smaller,		% Print acronym in smaller fontsize (required package: relsize)
%	printonlyused, % Print only acronyms that are actually used in the text
%%	dua, % print full names everytime
%%	nolist, % no list with all acronyms
%%	nohyperlinks, % no hyperlinks (required package: hyperref)
%]{acronym}

%%% Abkürzungsverzeichnis (Glossar) Neues Paket (kann nomencl und acronym ersetzen)
% muss nach hyperref eingebunden werden, um das Paket zu nutzen
% Abkürzungen werden nur im Glossar angezeigt, wenn sie im Dokument mindestens einmal genutzt wurden
\usepackage[
%	style=long,
	toc, % Glossar erscheint im Inhaltsverzeichnis
	acronym, % Setzt Akronyme in ein gesondertes Verzeichnis
%	footnote, % Setzt eine Fußnote beim ersten verwendet wird
%	nomain,
%	style=altlist,
	nopostdot, % löscht den schlusspunkt nach jeder description
]{glossaries}
\makeglossaries % Glossar generieren

\renewcommand\UrlFont{\color{black}\rmfamily\itshape}

\renewcommand{\familydefault}{\rmdefault}
\newcommand{\bflabel}[1]{\normalfont{\normalsize{#1}}\hfill}

%% Definition for Codeschnipsel im Fließtext
\newcommand{\code}{\texttt}

%% Todos mithilfe eines Rahmens hervorheben
\newcommand{\todo}[1]{\fbox{\parbox{\textwidth}{\textbf{To do:} #1}}}

\newacronym{url}{URL}{Uniform Resource Locator}
\newacronym{css}{CSS}{Cascading Style Sheets}
\newacronym{mituni}{MIT}{Massachusetts Institute of Technology}
\newacronym{abk}{Abk}{Abkürzung}
\newacronym{Abk}{Abk}{Abkürzung}

\newglossaryentry{pi}{
name=$\pi$,
description={Die Kreiszahl},
sort=Pi
}

%% figure and table names
\renewcaptionname{ngerman}{\figurename}{Abb.}
\renewcaptionname{english}{\figurename}{Fig.}
\renewcaptionname{ngerman,english}{\tablename}{Tab.}

