
%
%% Document Class (Koma Script) -----------------------------------------
%% Doc: scrguien.pdf
\documentclass[%
   %draft=true,     % draft mode (no images, layout errors shown)
   draft=false,     % final mode 
%%% --- Paper Settings ---
   paper=a4,
   paper=portrait, % landscape
   pagesize=auto, % driver
%%% --- Base Font Size ---
   fontsize=12pt,%
%%% --- Koma Script Version ---
   version=last, %
%%% --- Global Package Options ---
   ngerman, % language (passed to babel and other packages)
   parskip,
   numbers=noenddot,
   bibliography=totoc
]{scrreprt} % Classes: scrartcl, scrreprt, scrbook\usepackage[ngerman]{babel}

\usepackage[nottoc]{tocbibind} % Anzeigen des Literaturverzeichnisses im TOC
\usepackage{epsfig}
%\usepackage{times}
\usepackage{babel}
\usepackage{supertabular}
\usepackage{wrapfig}
\usepackage{multirow}
\usepackage[onehalfspacing]{setspace}
\usepackage{listings}
\usepackage{mathptmx}
\usepackage{geometry}
\usepackage{helvet}
\usepackage{courier}
\usepackage{setspace}
\usepackage{textcomp}
\usepackage[T1]{fontenc}
\usepackage[utf8]{inputenc}
\usepackage{float} % Notwendig fuer figure[h]
\usepackage[german=quotes]{csquotes}
\usepackage[style=alphabetic]{biblatex} % alternative: iso-authoryear
\usepackage{pdfpages}
% Fuer Schriftart Arial
%\usepackage[scaled]{uarial}


\lstset{ %
  backgroundcolor=\color{white},   % choose the background color; you must add \usepackage{color} or \usepackage{xcolor}
  basicstyle=\footnotesize,        % the size of the fonts that are used for the code
  breakatwhitespace=false,         % sets if automatic breaks should only happen at whitespace
  breaklines=true,                 % sets automatic line breaking
  captionpos=b,                    % sets the caption-position to bottom
%  commentstyle=\color{mygreen},    % comment style
  deletekeywords={...},            % if you want to delete keywords from the given language
  escapeinside={\%*}{*)},          % if you want to add LaTeX within your code
  extendedchars=true,              % lets you use non-ASCII characters; for 8-bits encodings only, does not work with UTF-8
 % frame=single,                    % adds a frame around the code
  keepspaces=true,                 % keeps spaces in text, useful for keeping indentation of code (possibly needs columns=flexible)
  keywordstyle=\color{blue},       % keyword style
  language=Octave,                 % the language of the code
  morekeywords={*,...},            % if you want to add more keywords to the set
  numbers=left,                    % where to put the line-numbers; possible values are (none, left, right)
  numbersep=5pt,                   % how far the line-numbers are from the code
%  numberstyle=\tiny\color{mygray}, % the style that is used for the line-numbers
  rulecolor=\color{black},         % if not set, the frame-color may be changed on line-breaks within not-black text (e.g. comments (green here))
  showspaces=false,                % show spaces everywhere adding particular underscores; it overrides 'showstringspaces'
  showstringspaces=true,          % underline spaces within strings only
  showtabs=true,                  % show tabs within strings adding particular underscores
  stepnumber=1,                    % the step between two line-numbers. If it's 1, each line will be numbered
%  stringstyle=\color{mymauve},     % string literal style
  tabsize=2,                       % sets default tabsize to 2 spaces
  title=\lstname,                   % show the filename of files included with \lstinputlisting; also try caption instead of title
  belowskip= 0pt 
}


\makeatother

\geometry{a4paper, left=45mm, right=20mm, top=30mm, bottom=30mm}

\renewcommand*{\chapterheadstartvskip}{\vspace*{0\baselineskip}}

\pagenumbering{roman}


\usepackage[automark,headsepline]{scrlayer-scrpage}

\clearpairofpagestyles
\cfoot[\pagemark]{\pagemark}
\lehead{\headmark}
\rohead{\headmark}

\pagestyle{scrheadings}


